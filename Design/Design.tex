\section{Design}
\label{sec:Design}

\subsection{Overview} 
This section discusses the design of the project including technologies used, classes, database schema, and user interface.

\subsection{Technologies}
This project uses a combination of ASP.NET Core MVC and Unity Virtual Reality. The Course Creator was developed with the former and the Virtual Reality Orienteering was developed with the latter. ASP.NET Core MVC ``is a lightweight, open source, highly testable presentation framework optimized for use with ASP.NET Core." \cite{dotnetcoremvc} ASP.NET Core is the underlying framework that enables development. ASP.NET Core ``is a cross-platform, high-performance, open-source framework for building modern, cloud-enabled, Internet-connected apps."\cite{dotnetcore}.\\
\\
The MVC in ASP.NET Core MVC stands for Model-View-Controller. MVC is an architectural pattern which separates an application into three main components: Models, Views, and Controllers. This separation helps achieve a ``separation of concerns", which asserts ``that software should be separated based on the kinds of work it performs" \cite{separationOfConcerns}. Models represent the data structure, independent of the user interface. It's responsible for data, logic, and rules of the application. Views represent the user interface and information. Views are responsible for presenting content with minimal logic. Controllers represent the logic and actions for models and views. Controllers are responsible for responding to user input and preforming operations. In summary, models are what it is, views are for what it looks like, and controllers are for how it behaves. 

\subsection{Point 2}
This gives Point 2

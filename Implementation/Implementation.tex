\section{Implementation}																	
\label{sec:Implementation}

\subsection{Overview} 
This section goes into details on the implementation of the Course Creator and Virtual Reality Orienteering programs. Key examples and general explanations are presented, to help the reader understand the code without rehashing the entire project. 

\subsection{Course Creator}
The Course Creator's primary responsibilities is creating courses and viewing results of those courses. Courses are comprised of locations, questions, and answers; each course has many locations, each location has many questions, and each question has many answers. 

\subsubsection{Authorization}
The Course Creator uses ASP.NET Core Identity to secure the application. This tool makes creating the registration and login pages simple, and has built-in security for securely storing passwords and other personal user data. ASP.NET Core Identity also makes use of user roles to ensure the user is authorized to do certain actions. The verified user role is an example of this in the project. Using a filter on the protected controllers is simple with Identity.
\begin{lstlisting}[caption=Securing Controllers using Filter on User Role, label=lst:FilterUserRole]
using Microsoft.AspNetCore.Authorization;

[Authorize(Roles = "Verified")]
public class HomeController : Controller
{
	...
}
\end{lstlisting} 
\subsubsection{Courses}
The course is the high level object with which the locations, questions, and answers become associated with. A course can be added, updated, or deleted. The \lstinline{CoursesController} handles all of these actions. Each of these actions has a corresponding View Razor page. The basic course model looks like the following:
\begin{lstlisting}[caption=Course Model, label=lst:CourseModel]
public class Course
{
	public int CourseId { get; set; }
	
	[DisplayName("Name")]
	public string Name { get; set; }

}
\end{lstlisting}
 
 The create and edit pages are similar to the other pages for the locations, questions, and answers, so this section will not go in depth on them, but instead elaborate further on the details and delete pages. The details page displays not only the course model information, but also all information relating to the course. To make the related course information modularize and reusable, the developer made use of ``View Components". View Components render a chunk of HTML output within another markup's file. This breaks up large markup files into smaller parts, reduces duplication of markup content, and provides an opportunity to use logic to control the rendered HTML. Each of the dependent classes on a course has a corresponding View Component which lists the data relating to the course. This can be seen in the course details page:
 \begin{lstlisting}[caption=Course Details Razor Page, label=lst:CourseDetails]
 @model ImmersiveQuiz.Models.Course
 
 @{
 ViewData["Title"] = "Details";
 }
 
 <h1>Details</h1>
 
 <div>
	 <h4>Course</h4>
	 <hr />
	 <dl class="row">
		 <dt class="col-sm-2">
			 @Html.DisplayNameFor(model => model.Name)
		 </dt>
		 <dd class="col-sm-10">
			 @Html.DisplayFor(model => model.Name)
		 </dd>
	 </dl>
 </div>
 <div class="mb-4">
	 <a class="badge large-badge bg-secondary text-light" asp-action="Index">Back to List</a> |
	 <a class="badge large-badge bg-success text-light" asp-controller="Scores" asp-action="Index" asp-route-id="@Model.CourseId">Scores</a>
 </div>
 <div>
	 <a class="badge large-badge bg-success text-light" asp-action="Create" asp-controller="Locations" asp-route-id="@Model.CourseId">Add Location</a> |
	 <a class="badge large-badge bg-primary text-light" asp-action="Edit" asp-route-id="@Model.CourseId">Edit</a>
	 <a class="badge large-badge bg-danger text-light" asp-action="Delete" asp-route-id="@Model.CourseId">Delete</a> 
 </div>
 @await Component.InvokeAsync("LocationList", new { CourseId = Model.CourseId.ToString(), search = "" })
 \end{lstlisting}
 
\subsubsection{Locations}

\subsubsection{Questions}

\subsubsection{Answers}

\subsubsection{Course Results}

\subsubsection{Web API}

\subsection{Virtual Reality Orienteering}
The Virtual Reality Orienteering primary responsibilities are to display the course and to keep track and submit the results. The Virtual Reality Orienteering program must have authorized communication with the Course Creator program.

\subsubsection{Authorization}

\subsubsection{Displaying Course}

\subsubsection{Tracking and Submitting Score Results}


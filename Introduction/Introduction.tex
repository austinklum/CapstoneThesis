\section{Introduction}
\label{sec:Introduction}

\subsection{Overview} 
The rise of globalism has prompted people of different cultures to increasingly work together and interact with one another. Thus, understanding other cultures and languages will become ever more important. Often times this can be hard to teach, especially in a classroom. Virtual reality can be used as a means to bridge the gap between real-world understanding and classroom knowledge. Virtual reality allows for a more immersive experience. A more immersive experience is a more effective way to engage students and promote learning. \\
\\
In 2017-2018 there was an initial virtual reality project conducted by Claire Mitchell to take tours of Medellin, Colombia. This project was a success and discussions were made to expand on this initial success. In 2019, there was a grant proposal for development of a new project to further enhance experiential learning. As virtual reality is a vanguard area of software development, such resources don't exist yet and would require new development. The proposal also requested an orienteering component to be included. Orienteering is an activity where participants \qquote{navigate between checkpoints along an unfamiliar course}{orienteering}. The primary purpose of adding an orienteering aspect is to add to the depth of cultural understanding, as orienteering requires the participants to have a more active role in the experience. \\
\\
The initial virtual reality project was predefined allowing no customizability within the courses. Also, there was no grading aspect of the project. The prior project helped expand students cultural understanding, but there was no built-in grading. The virtual reality wasn't very immersive and required holding a phone in front of the head to look around. The goals of this project want to improve upon these limitations. The project should be customizable, gradeable, and immersive all while garnering student interest and understanding of culture.  \\
\\ 
The project was designed with two components: A web application and a virtual reality application. The Course Creator web application is used to create the courses that the Virtual Reality Orienteering application uses to consume and display. The Course Creator program allows for full customizability of the orienteering courses that student's can take with verified users being able to create, edit and delete the course contents comprising of locations, questions, and answers. The Virtual Reality Orienteering program communicates with the Course Creator program to provide an immersive virtual reality experience. This immersive experience will help engage students while also giving clear metrics on understanding of material. Once a course has been completed by a student the graded results are found within the Course Creator program. As the courses created are custom tailored, the courses contain pertinent and applicable content based on the class being taught and the cultural context the class instructor hopes to convey. 
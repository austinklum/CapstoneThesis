\section{Testing}																	
\label{sec:Testing}

\subsection{Overview} 
This section describes the testing done to verify and validate that the Course Creator and Virtual Reality Orienteering programs are correct and can handle invalid or malicious input. As part of the iterative design for agile, the programs were tested as new functionality was added. 

\subsection{Input Validation}
Through a variety of means the programs ensure the input entered by users is valid. By preventing invalid input, the programs ensure the data received and presented is correct and avoids issues of unreliable data causing issues downstream. ASP.NET Core MVC provides Tag Helpers which ensures client-side validation takes place. These Tag Helpers use the Data Annotations on the models to determine valid input. A helpful error message is returned to the Razor Page views which describes to the user the invalid input. On the server-side, the Data Annotations on the model are also checked again. This prevents the user from bypassing the client-side and forcefully entered invalid data. When the Data Annotations are not sufficient, the developer manually checked input to ensure valid data. The developer grouped the validation into separate static classes to promote reuseability and modularization. \\
\\
One common type of attack on web platforms is Cross Side Scripting (XSS). XSS is inserting malicious code into the web page. Once this malicious code runs, the attacker can do anything within the web page to comprise the interactions the victim's has with the page. ASP.NET Core MVC automatically sanitizes input which is the process of disallowing, escaping, or preventing potential code from being executed. Any code entered via inputs is sanitized and cannot execute. Instead the worst case scenario is malicious code is rendered as HTML text on the page.\\
\\
Another common type of attack is SQL Injection. SQL Injection is inserting malicious SQL statements that will run against the database to either gain information or destroy data. Dynamically generated SQL has more avenues for injection of malicious SQL statements. As Entity Framework Core is how the database connection is created and SQL statements are executed, the only potential vectors for SQL Injection are the values asked for. The risk for SQL Injection is greatly mitigated, as these malicious values are prevented client-side, server-side, and, if not prevented by these other measures, sanitized. 
\subsection{Unit Testing}
Unit Testing is writing code to test your code. Unit Testing are automated tests that test the smallest piece of code. By ensuring all the individual parts of the code are correct, the entire program can be tested quickly, easily, and reproducibly. Unit Tests were created using xUnit. \qquote{xUnit.net is a free, open source, community-focused unit testing tool for the .NET Framework.}{xUnit}. One guiding principle for creating unit tests is to ensure reproducible, deterministic tests. Unit tests should avoid external dependencies or rely on other sources in order to determine the result. Unit tests are best used for testing business logic, ensuring expected input and output correctly do the data manipulation needed. Sometimes external dependencies cannot be avoided. This problem can be solved by ``mocking" the dependency and returning a predetermined result. In xUnit the common tool to do this mocking is called, Moq. Moq is a popular lightweight mocking library. Moq also supports LINQ to succinctly mock up dependencies. Unit tests are usually of the following format: Arrange, Act, and Assert. Arrange means setting up the test and mocking any dependencies, Act means executing the code piece that's being tested, and Assert means the output of the code matches the expected output. An example of a unit test from the project can be seen in Listing \ref{lst:UnitTest} using xUnit and Moq.
\begin{lstlisting}[caption=Example Unit Test,label=lst:UnitTest]
public class WebApiTests
{
	private readonly ImmersiveQuizApi _webApi;
	private readonly Mock<CourseContext> _mockCourseContext;
	private readonly Mock<LocationContext> _mockLocationContext;
	private readonly Mock<QuestionContext> _mockQuestionContext;
	private readonly Mock<AnswerContext> _mockAnswerContext;
	private readonly Mock<ScoreContext> _mockScoreContext;
	private readonly Mock<DbContextOptionsBuilder> _mockOptions;
	
	public WebApiTests()
	{
		_mockOptions = new Mock<DbContextOptionsBuilder>();
		_mockCourseContext = new Mock<CourseContext>(_mockOptions.Object);
		_mockLocationContext = new Mock<LocationContext>(_mockOptions.Object);
		_mockQuestionContext = new Mock<QuestionContext>(_mockOptions.Object);
		_mockAnswerContext = new Mock<AnswerContext>(_mockOptions.Object);
		_mockScoreContext = new Mock<ScoreContext>(_mockOptions.Object);
		
		_webApi = new ImmersiveQuizApi(_mockCourseContext.Object, _mockLocationContext.Object, _mockQuestionContext.Object, _mockAnswerContext.Object, _mockScoreContext.Object);
	}
	
	[Fact]
	public async Task Post_InvalidScore_ReturnsBadRequest()
	{
		// Arrange
		Score score = new Score()
		{
			StudentId = "1234",
			CourseId = 1,
			TimeScore = -1,
			PointScore = -1
		};
		
		// Act
		var response = await _webApi.SubmitScore(score);
		
		// Assert
		var badRequestResult = Assert.IsType<BadRequestObjectResult>(response);
		Assert.IsType<string>(badRequestResult.Value);
	} 
	
	...
	
}
\end{lstlisting}

\subsection{Acceptance Testing}
Acceptance Testing is testing to make sure the programs can preform the functionality that is required. Throughout the iterative agile process, functionality was tested. This means testing the programs as a user would, creating accounts, verifying users, creating courses with location, answers, and questions, etc. Along with this invalid input was tested to ensure a proper error message would appear. Dr. Mitchell, the project sponsor, also approved early prototypes of the Course Creator and Virtual Reality Orienteering program.  As functionality was tested as it was developed, bugs and defect were caught early and often. This reduced the amount of time needed testing everything at the end of the project.

\subsection{Integration Testing}
Integration Testing is a type of testing that ensure individual components work correctly together as a whole unit. This was primarily seen in the integration and communication of the Course Creator and Virtual Reality Orienteering programs. Ensuring that the Virtual Reality Orienteering program can interface with the Web API, and that the Web API could properly handle these requests, was vital. Without the testing of this area, the project could not function correctly. 
\section{Conclusion}																	
\label{sec:Conclusion}

\subsection{Overview} 
This project succeeds in achieving the goal of creating a customizable, gradeable, and immersive cultural virtual reality orienteering program. The two programs working in conjunction provide all the functional requirements expected. In addition, the non-functional requirements of being open-sourced, secure, portable, efficient, and easy-to-use have also been kept in mind during development.

\subsection{Challenges}
There were numerous challenges throughout the project. The biggest was the time and effort the developer had available. The developer also worked full-time as a Software Developer II at Kwik Trip. While at times related work as a software developer helped the development of the project, most often the work was unrelated and was a hindrance in terms of time and energy. This set the initial expect completion of Spring 2021, to almost a year and half later, with the actual completion during Summer 2022. \\
\\
Another challenge was the decreased involvement over time of the project sponsor. The project sponsor had a busy workload with other priorities and could not meet often past the initial requirements gathering and early demonstrations. This proved to be challenging as the primary stakeholder's feedback was not taken into consideration as often as an agile process dictates.\\
\\ 
As the developer was familiar with ASP.NET Core MVC from working experience, the Course Creator program didn't encounter too many hard blockers. The Virtual Reality Orienteering program on the other hand had many hard stops. The developer had no prior experience working with Unity or designing a VR program. The program flow in a game setting is very different than that of a website. The developer had to learn not only how to develop in Unity, but also understand the order of events and processes that enable real-time interaction within the worldspace. In addition, VR development is still not nearly as standardized or mature as web development. Several techniques or solutions tried proved to be ineffective or considered deprecated.   The UI interaction paired with looking around the worldspace took multiple attempts to get right and countless hours to correctly execute. At one point, the developer posted an online listing to pay for a consultant to help solve this issue. Luckily, the developer was able to figure out a solution without the help from a third party. 

\subsection{Future Work}
While the Course Creator and Virtual Reality Orienteering programs achieve the goals for this project, there are multiple areas that could be improved upon, extended, or enhanced. Currently, the programs are portable in the sense they are lightweight and simple to setup. One area of improvement would be to setup the Course Creator to a dedicated server that would be accessible on the UWL campus intranet. The Course Creator was developed as a web application, yet only exists on machines with the program executable downloaded. This would allow courses to persist outside of a local database and allow verified users to view, create, update, and delete data anywhere with a secured connection to UWL. Depending on how the server(s) are setup this would increase redundancy which would provide a more reliable and consistent experience. Database backups and contingencies could be enacted further protecting any loss of data.\\
\\
The Virtual Reality Orienteering program would also do well for rewriting and increasing modularization. Robert ``Uncle Bob" Martin is a famous software engineer and coding author. Five of Martin's principle's have become known as SOLID programming practices. SOLID stands for: Single-responsibility principle, Open-closed principle, Liskov Substitution Principle, Interface Segregation Principle, and Dependency Inversion principle. Over time the \lstinline{VRInputModule} grew in size and became a bit of an anti-pattern known as the ``God Class".  God classes break the `S' in SOLID programming practices which states \qquote{Classes should have one responsibility — one reason to change.}{srp} During development early mistakes were made and kluged together over time. To reapproach and refactor the program to be ``cleaner" and more maintainable would be great future work.\\
\\
Along with increasing modularization more ease-of-use features could also be added. Currently the course list view displays all courses in the order of creation. One useful feature would be adding a search functionality to the page so a verified user could easily search for the desired course. Another, ease-of-use feature would be adding a way to clone or copy content, whether that content be a course, location, question, or answer. This feature would be helpful in creating content that is to be reused by another course. Another feature that would aid in course creation would be a way to order the locations, questions, and answers. The current design displays to the student in the order the content was created. Perhaps it would be useful to change the order or to display these to the student in the Virtual Reality Orienteering program in a randomized order. This would ensure the content is displayed as the verified user wanted or to have each course displayed in a different order to prevent students from gaining prior knowledge or cheating.

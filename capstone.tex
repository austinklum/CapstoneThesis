\documentclass[letterpaper,12pt]{report}
%\documentclass[conference]{IEEEtran}
\usepackage{cite}
\usepackage[top=1in, bottom=1in, left=1in, right=1in]{geometry}
\usepackage{fancyhdr,tocloft,url,graphicx,float,listings,sidecap,wrapfig}
\usepackage[font=small,labelfont=bf,labelsep=period]{caption}
\usepackage{tabularx}
\usepackage{setspace}
\usepackage{hyperref}
\usepackage[all]{hypcap}
\usepackage{qtree,algorithm,algorithmic}
\pagestyle{plain}
\fancyhf{}
\lhead{}
\chead{}
\rhead{}
\cfoot{\thepage}


%\floatstyle{boxed}
%\restylefloat{figure}

\hypersetup {
	colorlinks=false,
	pdfborder={0 0 0},
}

\setcounter{secnumdepth}{3}
\renewcommand*\thesection{\arabic{section}.}
\renewcommand*\thesubsection{\thesection \arabic{subsection}.}
\renewcommand*\thesubsubsection{\thesubsection \arabic{subsubsection}.}

\setcounter{tocdepth}{3}
\renewcommand\contentsname{}				% TOC title
\renewcommand\listfigurename{}				% LOF title
\renewcommand\listtablename{}				% LOT title

\setlength\cftaftertoctitleskip{-0.5in}
\setlength\cftafterloftitleskip{-0.5in}
\setlength\cftafterlottitleskip{-0.5in}

%\renewcommand\bibsection{\section{References}}

% Adds C# Syntax highlighting for listing
% https://tex.stackexchange.com/questions/124953/syntax-highlighting-in-listings-for-c-that-it-looks-like-in-visual-studio
% http://blog.adnanmasood.com/2014/02/02/lyxlatex-formatting-for-the-c-code/

\usepackage{color}
\usepackage{listings}

\lstloadlanguages{% Check Dokumentation for further languages ...
	C,
	C++,
	csh,
	Java
}

\definecolor{red}{rgb}{0.6,0,0} % for strings
\definecolor{blue}{rgb}{0,0,0.6}
\definecolor{green}{rgb}{0,0.8,0}
\definecolor{cyan}{rgb}{0.0,0.6,0.6}

\lstset{
	language=csh,
	basicstyle=\footnotesize\ttfamily,
	numbers=left,
	numberstyle=\tiny,
	numbersep=5pt,
	tabsize=2,
	extendedchars=true,
	breaklines=true,
	frame=b,
	stringstyle=\color{blue}\ttfamily,
	showspaces=false,
	showtabs=false,
	xleftmargin=17pt,
	framexleftmargin=17pt,
	framexrightmargin=5pt,
	framexbottommargin=4pt,
	commentstyle=\color{green},
	morecomment=[l]{//}, %use comment-line-style!
	morecomment=[s]{/*}{*/}, %for multiline comments
	showstringspaces=false,
	morekeywords={ abstract, event, new, struct,
		as, explicit, null, switch,
		base, extern, object, this,
		bool, false, operator, throw,
		break, finally, out, true,
		byte, fixed, override, try,
		case, float, params, typeof,
		catch, for, private, uint,
		char, foreach, protected, ulong,
		checked, goto, public, unchecked,
		class, if, readonly, unsafe,
		const, implicit, ref, ushort,
		continue, in, return, using,
		decimal, int, sbyte, virtual,
		default, interface, sealed, volatile,
		delegate, internal, short, void,
		do, is, sizeof, while,
		double, lock, stackalloc,
		else, long, static,
		enum, namespace, string},
	keywordstyle=\color{cyan},
	identifierstyle=\color{red},
}
\usepackage{caption}
\DeclareCaptionFont{white}{\color{white}}
\DeclareCaptionFormat{listing}{\colorbox{blue}{\parbox{\textwidth}{\hspace{15pt}#1#2#3}}}
\captionsetup[lstlisting]{format=listing,labelfont=white,textfont=white, singlelinecheck=false, margin=0pt, font={bf,footnotesize}}
\renewcommand*{\lstlistlistingname}{List of Code Listings}
% End of C# hightlighting

\newcommand{\C}{C\# }
\newcommand{\qquote}[2]{``#1" \cite{#2}}

\begin{document}

\hypersetup{pageanchor=false}
%%  COVER PAGE
\begin{titlepage}
	\begin{center}
		\vspace*{0.5in}
		\begin{doublespace}
			\LARGE \textbf{Immersive Quiz for Spanish Learners} \\
			\vspace*{1in}
			\normalsize
			A Manuscript \\
			Submitted to \\
			the Department of Computer Science \\
			and the Faculty of the\\
			University of Wisconsin--La Crosse \\
			La Crosse, Wisconsin \\
			\vspace*{0.5in}
			by \\
			\large
			\textbf{Austin Klum} \\

			\vspace*{0.5in}
			\normalsize
			in Partial Fulfillment of the \\
			Requirements for the Degree of\\
			\Large{\textbf{Master of Software Engineering}} \\
			\normalsize
			May, 2022
		\end{doublespace}
	\end{center}
\end{titlepage}
	
\clearpage

%% SIGNATURE PAGE
\thispagestyle{empty}
\vspace*{0.3in}
\begin{center}
	\large{\textbf{Immersive Quiz for Spanish Learners}} \\ 
	\vspace{0.75in}
	\normalsize{By Austin Klum}
\end{center}

\vspace{0.5in}
\noindent We recommend acceptance of this manuscript in partial fulfillment of this candidate's requirements for the degree of Master of Software Engineering in Computer Science. The candidate has completed the oral examination requirement of the capstone project for the degree. \\

\noindent
\begin{tabularx}{\textwidth}{p{3in}Xp{2in}}
		\rule{0pt}{50pt} & & \\
	\hrulefill & & \hrulefill \\
	Prof. Elliott Forbes & & Date \\
	Examination Committee Member & & \\
	\rule{0pt}{50pt} & & \\
	\hrulefill & & \hrulefill \\
	Prof. Thomas Gendreau & & Date \\
	Examination Committee Member & & \\
	\rule{0pt}{50pt} & & \\
	\hrulefill & & \hrulefill \\
	Prof. Dipankar Mitra & & Date \\
	Examination Committee Member & & \\
\end{tabularx}


\clearpage

\hypersetup{pageanchor=true}
\setcounter{page}{1}
\pagenumbering{roman}
\renewcommand\arraystretch{1.5}
\section*{Abstract}
\addcontentsline{toc}{section}{Abstract}
Austin Klum, J., ``Immersive Quiz for Spanish Learners,'' Master of Software Engineering, May 2022, (Elliot Forbes, Ph.D.). \\


%% ABSTRACT
This manuscript describes the development of a quiz creation tool combined with a virtual reality component to provide an immersive quiz taking experience for Spanish learners. The quizzes also have an orienteering course aspect as well, where each quiz is comprised of multiple timed locations where all questions must be completed correctly before continuing onto the next location. 
\clearpage

%%% ACKNOWLEDGEMENTS
\section*{Acknowledgments}
\addcontentsline{toc}{section}{Acknowledgments}
I would like to express my thanks to the Department of Computer Science at the University of Wisconsin--La Crosse for providing the learning materials and computing environment for my project.
\clearpage

%% TABLE OF CONTENTS
\section*{Table of Contents}
\tableofcontents
\clearpage


%% LIST OF TABLES
\section*{List of Tables}
\addcontentsline{toc}{section}{List of Tables}
\listoftables
\clearpage

%% LIST OF FIGURES
\section*{List of Figures}
\addcontentsline{toc}{section}{List of Figures}
\listoffigures
\clearpage

%% LIST OF LISTINGS
\lstlistoflistings
\addcontentsline{toc}{section}{List of Code Listings}
\clearpage


%% GLOSSARY
\section*{Glossary}
\addcontentsline{toc}{section}{Glossary}
\subsection*{ASP.NET Core}

Lorem ipsum

\subsection*{ASP.NET Core MVC}
Lorem ipsum
\subsection*{Unity}
Lorem ipsum
\subsection*{Unity XR}
Lorem ipsum
\subsection*{\C}
Lorem ipsum
\subsection*{LINQ}
Lorem ipsum
\subsection*{Entity Framework Core}
Lorem ipsum
\subsection*{SQL Server}
Lorem ipsum
\subsection*{Transparent Date Encryption (TDE)}
Lorem ipsum
\subsection*{Basic Authentication}
Lorem ipsum
\subsection*{JSON}
Lorem ipsum
\subsection*{HTTPS}
Lorem ipsum
\subsection*{REST}
Lorem ipsum
 \subsection*{CSS}
Lorem ipsum
\subsection*{HTML}
Lorem ipsum
\subsection*{Bootstrap}
Lorem ipsum
\subsection*{Web API}
Lorem ipsum
\subsection*{GUID}
Lorem ipsum
\subsection*{Git}
Lorem ipsum
\subsection*{Dependency Injection}
Lorem ipsum
\subsection*{Razor View Engine}
Lorem ipsum
\subsection*{Visual Studio}
Lorem ipsum
\subsection*{Unity Editor}
Lorem ipsum


\clearpage

\setcounter{page}{1}
\pagenumbering{arabic}

\IfFileExists{Introduction/Introduction}{\section{Introduction}
\label{sec:Introduction}

\subsection{Overview} 
The rise of globalism has prompted people of different cultures to increasingly work together and interact with one another. Thus, understanding other cultures and languages will become ever more important. Often times this can be hard to teach, especially in a classroom. Virtual reality can be used as a means to bridge the gap between real-world understanding and classroom knowledge. Virtual reality allows for a more immersive experience. A more immersive experience is a more effective way to engage students and promote learning. \\
\\
In 2017-2018 there was an initial virtual reality project conducted by Claire Mitchell to take tours of Medellin, Colombia. This project was a success and discussions were made to expand on this initial success. In 2019, there was a grant proposal for development of a new project to further enhance experiential learning. As virtual reality is a vanguard area of software development, such resources don't exist yet and would require new development. The proposal also requested an orienteering component to be included. Orienteering is an activity where participants \qquote{navigate between checkpoints along an unfamiliar course}{orienteering}. The primary purpose of adding an orienteering aspect is to add to the depth of cultural understanding, as orienteering requires the participants to have a more active role in the experience. \\
\\
The initial virtual reality project was predefined allowing no customizability within the courses. Also, there was no grading aspect of the project. The prior project helped expand students cultural understanding, but there was no built-in grading. The virtual reality wasn't very immersive and required holding a phone in front of the head to look around. The goals of this project want to improve upon these limitations. The project should be customizable, gradeable, and immersive all while garnering student interest and understanding of culture.  
}{}\clearpage
\IfFileExists{Requirements/Requirements}{\section{Requirements}
\label{sec:Requirements}

\subsection{Overview} 
This gives a brief overview of this section.

\subsection{Point 1}
This subsection gives a great deal of precise description supporting point 1.  For example,


\subsection{Point 2}
This gives Point 2
}{}\clearpage
\IfFileExists{Design/Design}{\section{Design}
\label{sec:Design}

\subsection{Overview} 
This section discusses the design of the project including technologies used, classes, database schema, and user interface.

\subsection{Technologies}
This project uses a combination of ASP.NET Core MVC and Unity XR. The Course Creator was developed with the former and the Virtual Reality Orienteering was developed with the latter.

\subsubsection{Course Creator}
ASP.NET Core MVC \qquote{is a lightweight, open source, highly testable presentation framework optimized for use with ASP.NET Core.}{dotnetcoremvc} ASP.NET Core is the underlying framework that enables development. ASP.NET Core \qquote{is a cross-platform, high-performance, open-source framework for building modern, cloud-enabled, Internet-connected apps.}{dotnetcore}.\\
\\
The MVC in ASP.NET Core MVC stands for Model-View-Controller. MVC is an architectural pattern which separates an application into three main components: Models, Views, and Controllers. This separation helps achieve a ``separation of concerns", which asserts ``that software should be separated based on the kinds of work it performs" \cite{separationOfConcerns}. Models represent the data structure, independent of the user interface. Models are responsible for data, logic, and rules of the application. Views represent the user interface and information. Views are responsible for presenting content with minimal logic. Controllers represent the logic and actions for models and views. Controllers are responsible for responding to user input and preforming operations. In summary, models are what it is, views are for what it looks like, and controllers are for how it behaves. \\
\\
ASP.NET Core MVC was chosen because of the developer's prior experience and the extensive functionalities that the framework provides. ASP.NET Core MVC is open-source and multi-platform, supporting Windows, macOS, and Linux out of the box. ASP.NET Core MVC provides routing which is a useful URL-mapping component. Routing provides for easy link generation without regard to the actual file structure. ASP.NET Core MVC also provides model   binding on requests. This makes incoming and outgoing requests easy to process or generate without further processing. Model validation is also built-in using data annotation attributes. These are pre-built or custom attributes within the model that validate on the fly rather than requiring explicit checking. This allows for guaranteeing the state of the model before further processing. Dependency injection is also supported, which is a key feature for building the controllers. Dependency Injection (DI) is a software design pattern, \qquote{which is a technique for achieving Inversion of Control (IoC) between classes and their dependencies.}{dependencyInjection} A dependency is an object than another object depends on. When a class depends on another class, future changes become problematic. Dependency Injection solves this by using an interface to abstract the dependency, registering the dependency in a service container, and then injects the dependency when needed. ASP.NET Core MVC provides filters which can be placed on controllers so that all actions must meet this filter. Oftentimes filters are added for exception handling or authorization. This way instead of each action requiring authorization, one can require the entire controller with all actions to be authorized. ASP.NET Core MVC is also a great platform for building Web APIs. HTTP content-negotiation with common data formats such as JSON or XML is already supported. The Razor view engine is another key advantage for ASP.NET Core MVC. Razor view engine is a compact and easy template markup language used for defining views with embedded \C code. Razor can be used to dynamically generate web content with a mix of server side and client side code. Tag Helpers are also used with Razor to facilitate creating and rendering HTML. Tag Helpers bind to certain HTML elements and vastly improves their use cases. 

\subsubsection{Virtual Reality Orienteering}
Unity is a cross-platform game engine developed by Unity Technologies with the goal to provide developers with the tools to make game development simple. Unity supports a vast variety of platforms and user experiences. These tools extend to desktop, mobile, console, and virtual reality with support for 2D, 3D, and other experiences. It's also used in other areas outside of game development such as film, engineering, architecture, and automotive modeling. This makes Unity a popular choice, not only for the developer of this project, but for the world at large. \qquote{In the fourth quarter of 2021, Unity had, on average, 3.9 billion monthly active end users who consumed content created or operated with its solutions.}{unityNumbers} \\
\\
The basic components of a game developed in Unity are GameObjects, Assets, Scenes, and Scripts. Every object in a game is a GameObject. Assets are reusable items that can be used throughout the game. They can be of any file type that Unity supports, such as a 3D model, audio file, image. Oftentimes these assets come from outside Unity, created by the developer or other developers which are found in the Asset Store. Scenes contain the objects of the game. Oftentimes these are split into logical groupings such as main menu, individual levels, or the environment. Scripts are what controls the behavior of the GameObjects. Without scripts the game would be static and have no interaction or logic. Unity supports \C natively and is the standard used for scripting. A new script has two functions, \lstinline{Start()} and \lstinline{Update()}. \lstinline{Start()} is called once by Unity before gameplay begins which is used to setup initial configurations. \lstinline{Update()} is called once per frame update for the GameObject. This is used to handle anything that needs to be done over time in the gameplay, such as movement, triggering actions, or responding to user input.  

\begin{lstlisting}[caption=New Script in Unity]
using UnityEngine;
using System.Collections;

public class NewBehaviourScript : MonoBehaviour {

	// Use this for initialization
	void Start()
	{
	
	}
	
	// Update is called once per frame
	void Update() 
	{
	
	}
}
\end{lstlisting}

Unity XR Interaction Toolkit is a cross-platform plugin used for virtual, mixed, and augmented reality. XR Interaction Toolkit provides easy built-in functionality to select, grab, throw, rotate objects within a VR scene. This also extends to the UI interactions and the haptic feedback that comes with it. The ability to look around and move within the worldspace is also provided with this plugin. As this project uses a HTC Vive headset, OpenXR was the targeted development platform. OpenXR is an open standard that targets a wide array of virtual reality devices.

\subsection{Class Diagrams}

\subsubsection{Course Creator}

\subsubsection{Virtual Reality Orienteering}

\subsection{Database}
The database uses Microsoft's SQL Server. SQL Server is the de facto used for .NET projects.  SQL Server is a relation database management system which manages and stores information. The standard tool for working with SQL Server is SQL Server Management Studio which makes database changes and transactions easy. Transparent Data Encryption (TDE) encrypts data files at rest. This means that any data files stored are encrypted preventing any malicious attempts to read the database.

\subsubsection{Database Schema}

\subsection{Communication between Programs}
This project comprises of two separate programs which must communicate with each other, and do so securely. The Course Creator has a RESTful Web API which requires Basic Authentication to make calls to.  Basic Authentication requires a username and password over an HTTP connection. Using just an HTTP connection is not secure though, as the username and password are sent over in plaintext. This means any malicious user can sniff the network and easily obtain the credentials. To counter this the project requires a secure connection using HTTPS, with the `S' meaning secure. An HTTPS connection encrypts any data sent over it; thereby securing the Basic Authentication credentials. The RESTful Web API provides predefined endpoints for authorized users to use with REST calls. REST stands for REpresentational State Transfer and is a architectural standard for communication on the web. REST is stateless, meaning that the server does not need to know about what state the client is or vice versa. This allows for communication without needing to know the previous messages. A REST request usually is comprised of an HTTP verb, header, URI path, and an optional body containing data. The four basic HTTP verbs are GET, POST, PUT, and DELETE. GET retrieves data or a specific resource. POST creates new data or a new resource. PUT updates data or a specific resource. Finally, DELETE removes data or a specific resource. The header is used for carrying pertinent information about the request being made. This includes the authorization, cookies, caching, and other logistical information for fulfilling the request. The REST requests consume or return data in JSON which is a standard data-interchange format. JSON is easy for humans to read and simple for computers to parse and generate.


}{}\clearpage
\IfFileExists{Testing/Testing}{\section{Testing}																	
\label{sec:Testing}

\subsection{Overview} 
This section describes the testing done to verify and validate that the Course Creator and Virtual Reality Orienteering programs are correct and can handle invalid or malicious input. As part of the iterative design for agile, the programs were tested as new functionality was added. 

\subsection{Input Validation}
Through a variety of means the programs ensure the input entered by users is valid. By preventing invalid input, the programs ensure the data received and presented is correct and avoids issues of unreliable data causing issues downstream. ASP.NET Core MVC provides Tag Helpers which ensures client-side validation takes place. These Tag Helpers use the Data Annotations on the models to determine valid input. A helpful error message is returned to the Razor Page views which describes to the user the invalid input. On the server-side, the Data Annotations on the model are also checked again. This prevents the user from bypassing the client-side and forcefully entered invalid data. When the Data Annotations are not sufficient, the developer manually checked input to ensure valid data. The developer grouped the validation into separate static classes to promote reuseability and modularization. \\
\\
One common type of attack on web platforms is Cross Side Scripting (XSS). XSS is inserting malicious code into the web page. Once this malicious code runs, the attacker can do anything within the web page to comprise the interactions the victim's has with the page. ASP.NET Core MVC automatically sanitizes input which is the process of disallowing, escaping, or preventing potential code from being executed. Any code entered via inputs is sanitized and cannot execute. Instead the worst case scenario is malicious code is rendered as HTML text on the page.\\
\\
Another common type of attack is SQL Injection. SQL Injection is inserting malicious SQL statements that will run against the database to either gain information or destroy data. Dynamically generated SQL has more avenues for injection of malicious SQL statements. As Entity Framework Core is how the database connection is created and SQL statements are executed, the only potential vectors for SQL Injection are the values asked for. The risk for SQL Injection is greatly mitigated, as these malicious values are prevented client-side, server-side, and, if not prevented by these other measures, sanitized. 
\subsection{Unit Testing}
Unit Testing is writing code to test your code. Unit Testing are automated tests that test the smallest piece of code. By ensuring all the individual parts of the code are correct, the entire program can be tested quickly, easily, and reproducibly. Unit Tests were created using xUnit. \qquote{xUnit.net is a free, open source, community-focused unit testing tool for the .NET Framework.}{xUnit}. One guiding principle for creating unit tests is to ensure reproducible, deterministic tests. Unit tests should avoid external dependencies or rely on other sources in order to determine the result. Unit tests are best used for testing business logic, ensuring expected input and output correctly do the data manipulation needed. Sometimes external dependencies cannot be avoided. This problem can be solved by ``mocking" the dependency and returning a predetermined result. In xUnit the common tool to do this mocking is called, Moq. Moq is a popular lightweight mocking library. Moq also supports LINQ to succinctly mock up dependencies. Unit tests are usually of the following format: Arrange, Act, and Assert. Arrange means setting up the test and mocking any dependencies, Act means executing the code piece that's being tested, and Assert means the output of the code matches the expected output. An example of a unit test from the project can be seen in Listing \ref{lst:UnitTest} using xUnit and Moq.
\begin{lstlisting}[caption=Example Unit Test,label=lst:UnitTest]
public class WebApiTests
{
	private readonly ImmersiveQuizApi _webApi;
	private readonly Mock<CourseContext> _mockCourseContext;
	private readonly Mock<LocationContext> _mockLocationContext;
	private readonly Mock<QuestionContext> _mockQuestionContext;
	private readonly Mock<AnswerContext> _mockAnswerContext;
	private readonly Mock<ScoreContext> _mockScoreContext;
	private readonly Mock<DbContextOptionsBuilder> _mockOptions;
	
	public WebApiTests()
	{
		_mockOptions = new Mock<DbContextOptionsBuilder>();
		_mockCourseContext = new Mock<CourseContext>(_mockOptions.Object);
		_mockLocationContext = new Mock<LocationContext>(_mockOptions.Object);
		_mockQuestionContext = new Mock<QuestionContext>(_mockOptions.Object);
		_mockAnswerContext = new Mock<AnswerContext>(_mockOptions.Object);
		_mockScoreContext = new Mock<ScoreContext>(_mockOptions.Object);
		
		_webApi = new ImmersiveQuizApi(_mockCourseContext.Object, _mockLocationContext.Object, _mockQuestionContext.Object, _mockAnswerContext.Object, _mockScoreContext.Object);
	}
	
	[Fact]
	public async Task Post_InvalidScore_ReturnsBadRequest()
	{
		// Arrange
		Score score = new Score()
		{
			StudentId = "1234",
			CourseId = 1,
			TimeScore = -1,
			PointScore = -1
		};
		
		// Act
		var response = await _webApi.SubmitScore(score);
		
		// Assert
		var badRequestResult = Assert.IsType<BadRequestObjectResult>(response);
		Assert.IsType<string>(badRequestResult.Value);
	} 
	
	...
	
}
\end{lstlisting}

\subsection{Acceptance Testing}
Acceptance Testing is testing to make sure the programs can preform the functionality that is required. Throughout the iterative agile process, functionality was tested. This means testing the programs as a user would, creating accounts, verifying users, creating courses with location, answers, and questions, etc. Along with this invalid input was tested to ensure a proper error message would appear. Dr. Mitchell, the project sponsor, also approved early prototypes of the Course Creator and Virtual Reality Orienteering programs.  As functionality was tested as it was developed, bugs and defect were caught early and often. This reduced the amount of time needed testing everything at the end of the project.

\subsection{Integration Testing}
Integration Testing is a type of testing that ensure individual components work correctly together as a whole unit. This was primarily seen in the integration and communication of the Course Creator and Virtual Reality Orienteering programs. Ensuring that the Virtual Reality Orienteering program can interface with the Web API, and that the Web API could properly handle these requests, was vital. Without the testing of this area, the project could not function correctly. }{}\clearpage
\IfFileExists{Conclusion/Conclusion}{\section{Conclusion}																	
\label{sec:Conclusion}

\subsection{Overview} 
This project succeeds in achieving the goal of creating a customizable immersive cultural virtual reality orienteering program. The two programs working in conjunction provide all the functional requirements expected. In addition, the non-functional requirements of being open-sourced, secure, portable, efficient, and easy-to-use have also been kept in mind during development.

\subsection{Challenges}
There were numerous challenges throughout the project. The biggest was the time and effort the developer had available. The developer also worked full-time as a Software Developer II at Kwik Trip. While at times related work as a software developer helped the development of the project, most often the work was unrelated and was a hindrance in terms of time and energy. This set the initial expect completion of Spring 2021, to almost a year and half later, with the actual completion during Summer 2022. \\
\\
Another challenge was the decreased involvement over time of the project sponsor. The project sponsor had a busy workload with other priorities and could not meet often past the initial requirements gathering and early demonstrations. This proved to be challenging as the primary stakeholder's feedback was not taken into consideration as often as an agile process dictates.\\
\\ 
As the developer was familiar with ASP.NET Core MVC from working experience, the Course Creator program didn't encounter too many hard blockers. The Virtual Reality Orienteering program on the other hand had many hard stops. The developer had no prior experience working with Unity or designing a VR program. The program flow in a game setting is very different than that of a website. The developer had to learn not only how to develop in Unity, but also understand the order of events and processes that enable real-time interaction within the worldspace. In addition, VR development is still not nearly as standardized or mature as web development. Several techniques or solutions tried proved to be ineffective or considered deprecated.   The UI interaction paired with looking around the worldspace took multiple attempts to get right and countless hours to correctly execute. At one point, the developer posted an online listing to pay for a consultant to help solve this issue. Luckily, the developer was able to figure out a solution without the help from a third party. 

\subsection{Future Work}
While the Course Creator and Virtual Reality Orienteering programs achieve the goals for this project, there are multiple areas that could be improved upon, extended, or enhanced. Currently, the programs are portable in the sense they are lightweight and simple to setup. One area of improvement would be to setup the Course Creator to a dedicated server that would be accessible on the UWL campus intranet. The Course Creator was developed as a web application, yet only exists on machines with the program executable downloaded. This would allow courses to persist outside of a local database and allow verified users to view, create, update, and delete data anywhere with a secured connection to UWL. Depending on how the server(s) are setup this would increase redundancy which would provide a more reliable and consistent experience. Database backups and contingencies could be enacted further protecting any loss of data.\\
\\
The Virtual Reality Orienteering program would also do well for rewriting and increasing modularization. Robert ``Uncle Bob" Martin is a famous software engineer and coding author. Five of Martin's principle's have become known as SOLID programming practices. SOLID stands for: Single-responsibility principle, Open-closed principle, Liskov Substitution Principle, Interface Segregation Principle, and Dependency Inversion principle. Over time the \lstinline{VRInputModule} grew in size and became a bit of an anti-pattern known as the ``God Class".  God classes break the `S' in SOLID programming practices which states \qquote{Classes should have one responsibility — one reason to change.}{srp} During development early mistakes were made and kluged together over time. To reapproach and refactor the program to be ``cleaner" and more maintainable would be great future work.
}{}\clearpage

An article \cite{anarticle}\\
A book \cite{abook}\\
A series \cite{bookseries}\\
Someone's thesis \cite{thesis}\\
Some technical report \cite{report}\\
A collection \cite{collection}\\
Visited website \cite{website}\\
Accepted for publication \cite{acceptedpub}\\
Submitted for publication \cite{unpub}\\
Not published \cite{notpub}\\
Conversation \cite{conv}

%% BIBLIOGRAPHY
%\label{sec:bibliography}
\addcontentsline{toc}{section}{Bibliography}
\bibliographystyle{IEEEtran}
\bibliography{Bibliography}

\clearpage



\end{document}
